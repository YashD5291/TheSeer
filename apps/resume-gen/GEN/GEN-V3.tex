\documentclass[10pt, letterpaper]{article}

% ================================
% PACKAGES
% ================================
\usepackage[
    ignoreheadfoot,
    top=1.8cm,
    bottom=1.8cm,
    left=1.8cm,
    right=1.8cm,
    footskip=1.0cm,
]{geometry}

\usepackage{titlesec}
\usepackage{tabularx}
\usepackage{array}
\usepackage[dvipsnames]{xcolor}
\usepackage{enumitem}
\usepackage{fontawesome5}
\usepackage{amsmath}
\usepackage[
    pdfauthor={Juan Flores},
    colorlinks=true,
    urlcolor=black
]{hyperref}
\usepackage[pscoord]{eso-pic}
\usepackage{calc}
\usepackage{bookmark}
\usepackage{lastpage}
\usepackage{changepage}
\usepackage{paracol}
\usepackage{ifthen}
\usepackage{needspace}
\usepackage{iftex}

% ================================
% FONT CONFIGURATION
% ================================
\ifPDFTeX
    \input{glyphtounicode}
    \pdfgentounicode=1
    \usepackage[T1]{fontenc}
    \usepackage[utf8]{inputenc}
    \usepackage{lmodern}
    \usepackage{charter} % Professional font (Type1)
\else
    \usepackage{fontspec}
    \setmainfont{Charter} % Professional font (system TTF for XeTeX/tectonic)
\fi

% ================================
% DOCUMENT SETTINGS
% ================================
\pagestyle{empty} % No header or footer
\setcounter{secnumdepth}{0} % No section numbering
\setlength{\parindent}{0pt} % No paragraph indentation
\setlength{\topskip}{0pt}
\setlength{\columnsep}{0.15cm}
\pagenumbering{gobble}
\AtBeginEnvironment{adjustwidth}{\partopsep0pt}

% ================================
% SECTION FORMATTING
% ================================
\titleformat{\section}{
    \needspace{4\baselineskip}
    \bfseries
    \large
}{}{0pt}{}[\vspace{2pt}\titlerule]

\titlespacing{\section}{
    -1pt    % left space
}{
    0.3cm   % top space
}{
    0.25cm  % bottom space
}

% ================================
% CUSTOM BULLET POINTS
% ================================
\renewcommand\labelitemi{$\vcenter{\hbox{\small$\bullet$}}$}

% ================================
% CUSTOM ENVIRONMENTS
% ================================

% Environment for bullet points in experience
\newenvironment{highlights}{
    \begin{itemize}[
        topsep=0.10cm,
        parsep=0.10cm,
        partopsep=0pt,
        itemsep=0.05cm,
        leftmargin=20pt
    ]
}{
    \end{itemize}
}

% Environment for one column entries
\newenvironment{onecolentry}{
    \begin{adjustwidth}{0cm + 0.00001cm}{0cm + 0.00001cm}
}{
    \end{adjustwidth}
}

% Environment for two column entries (used in experience)
\newenvironment{twocolentry}[2][]{
    \onecolentry
    \def\secondColumn{#2}
    \setcolumnwidth{\fill, 4.5cm}
    \begin{paracol}{2}
}{
    \switchcolumn
    \raggedleft
    \secondColumn
    \end{paracol}
    \endonecolentry
}

% Environment for the header
\newenvironment{header}{
    \setlength{\topsep}{0pt}
    \par\kern\topsep
    \centering
    \linespread{1.2}
}{
    \par\kern\topsep
}

% ================================
% CUSTOM COMMANDS
% ================================
\newcommand{\experienceSeparator}{\vspace{0.35cm}} % Increased spacing between companies

% Save the original href command
\let\hrefWithoutArrow\href

% ================================
% DOCUMENT CONTENT
% ================================
\begin{document}

    % Header separator
    \newcommand{\AND}{\unskip
        \cleaders\copy\ANDbox\hskip\wd\ANDbox
        \ignorespaces
    }
    \newsavebox\ANDbox
    \sbox\ANDbox{$|$}

    % ================================
    % HEADER SECTION
    % ================================
    \begin{header}
        \fontsize{22pt}{22pt}\selectfont
        \textbf{Juan Flores}

        \vspace{6pt}

        \normalsize
        {\hrefWithoutArrow{mailto:juanflores.work@outlook.com}{juanflores.work@outlook.com}}
        {|}
        {\hrefWithoutArrow{tel:+1-704-363-9900}{+1 (704) 363-9900}}

        {\hrefWithoutArrow{https://linkedin.com/in/juan-flores-ml}{linkedin.com/in/juan-flores-ml}}
        {|}
        {\hrefWithoutArrow{https://github.com/juanf-0gravity}{github.com/juanf-0gravity}}
    \end{header}

    \vspace{0.3cm}

    \begin{onecolentry}
        Machine Learning Engineer with 7+ years building production systems and 3+ years shipping AI-powered product features end-to-end. Deep expertise in LLMs, prompt engineering, RAG, and evaluation methods, with a track record of iteratively improving GenAI features using quantitative and qualitative signals. Experienced supporting on-call production workloads and implementing observability (New Relic, Datadog). Proven cross-functional collaborator who thrives in ambiguity, shipping reliable AI features at Covariant (RFM-1) and Labelbox (Model Foundry). Driven by first principles thinking to deliver user-facing impact.
    \end{onecolentry}

    % ================================
    % SKILLS
    % ================================
    \section{Skills}

    \begin{onecolentry}
        \textbf{LLM \& GenAI:} Prompt Engineering, RAG, Evaluation Methods, RLHF, DPO, Knowledge Distillation, PEFT/LoRA
    \end{onecolentry}

    \vspace{0.15cm}

    \begin{onecolentry}
        \textbf{ML Frameworks:} PyTorch, JAX, Transformers, Hugging Face, Triton
    \end{onecolentry}

    \vspace{0.15cm}

    \begin{onecolentry}
        \textbf{Inference \& Serving:} vLLM, TensorRT-LLM, Triton Server, ONNX, BentoML, Ray Serve
    \end{onecolentry}

    \vspace{0.15cm}

    \begin{onecolentry}
        \textbf{Training \& Scale:} DeepSpeed, FSDP, Mixed Precision, Multi-GPU, Model Parallelism
    \end{onecolentry}

    \vspace{0.15cm}

    \begin{onecolentry}
        \textbf{Backend \& Data:} Python, Go, SQL, Kafka, Spark, Redis, Vector DBs (FAISS, Pinecone, Weaviate)
    \end{onecolentry}

    \vspace{0.15cm}

    \begin{onecolentry}
        \textbf{Production \& Observability:} New Relic, Datadog, Weights \& Biases, MLflow, Docker, Kubernetes, CI/CD, A/B Testing
    \end{onecolentry}

    % ================================
    % EXPERIENCE
    % ================================
    \section{Experience}

    % Company 1: Labelbox
    \begin{twocolentry}{
        Jun 2024 -- Oct 2025
    }
        \textbf{Senior AI Engineer} | Labelbox -- Remote (San Francisco, CA)
    \end{twocolentry}

    \vspace{0.10cm}
    \begin{onecolentry}
        \begin{highlights}
            \item Owned end-to-end development of Model Foundry's LLM-powered labeling features, from prototype to production, designing multi-model routing across GPT-4, Claude, and Gemini based on task type, cost, and latency requirements.
            \item Iterated on GenAI feature quality through RLHF loops using human preference ranking, implementing evaluation strategies that reduced hallucination rates by 40\% in production workflows for enterprise clients including Genentech and P\&G.
            \item Built evaluation infrastructure for multi-turn conversations and rubric-based assessment, enabling quantitative quality metrics (inter-rater reliability, agreement scores) that drove iterative model improvements.
            \item Collaborated cross-functionally with domain experts and product stakeholders to design AI-driven experiences, managing workforce quality systems that matched 500+ evaluators to specialized tasks based on expertise and performance signals.
            \item Optimized LLM inference with vLLM and TensorRT on A100 clusters, achieving sub-second responses while meeting production reliability and GDPR constraints for large-scale enterprise deployments.
        \end{highlights}
    \end{onecolentry}

    \experienceSeparator

    % Company 2: Covariant
    \begin{twocolentry}{
        Jan 2022 -- May 2024
    }
        \textbf{Machine Learning Engineer} | Covariant -- Remote (Berkeley, CA)
    \end{twocolentry}

    \vspace{0.10cm}
    \begin{onecolentry}
        \begin{highlights}
            \item Shipped RFM-1 (Robotics Foundation Model) end-to-end, an 8B-parameter multimodal transformer, leading training infrastructure on 128 A100 GPUs using PyTorch FSDP and collaborating across teams to deploy to production warehouse environments.
            \item Improved model reliability through RLHF fine-tuning on fleet-collected data from Otto Group and McKesson deployments, using evaluation loops with human preference feedback to achieve 20\% higher accuracy in zero-shot scenarios.
            \item Designed experiment frameworks for model evaluation, implementing in-context learning capabilities that enabled rapid adaptation to novel objects without retraining, validated through systematic A/B comparisons across deployment sites.
            \item Optimized production serving with Triton Server and TensorRT quantization, supporting on-call reliability for edge deployments processing real-time robotic decisions at sub-100ms latency.
            \item Developed physics prediction features using diffusion models, iterating through fast feedback loops with warehouse operators to align model behavior with user expectations via constitutional AI principles.
        \end{highlights}
    \end{onecolentry}

    \experienceSeparator

    % Company 3: Rockset
    \begin{twocolentry}{
        Jun 2020 -- Dec 2021
    }
        \textbf{Data Engineer} | Rockset (acquired by OpenAI 2024) -- Remote (San Mateo, CA)
    \end{twocolentry}

    \vspace{0.10cm}
    \begin{onecolentry}
        \begin{highlights}
            \item Built real-time data pipelines using Kafka and Spark for production analytics, enabling sub-millisecond querying through Converged Index architecture serving customers like JetBlue and Klarna.
            \item Integrated vector search capabilities using FAISS and HNSW graphs, implementing retrieval infrastructure that supported similarity queries on embeddings at scale.
            \item Contributed to schema inference systems handling dynamic JSON documents, supporting reliable analytics on semi-structured data from MongoDB CDC and DynamoDB streams.
            \item Supported production reliability through distributed query system troubleshooting and automated scaling, maintaining high uptime for customer-facing workloads. (Rockset acquired by OpenAI in 2024 for retrieval infrastructure.)
        \end{highlights}
    \end{onecolentry}

    \experienceSeparator

    % Company 4: AvidXchange
    \begin{twocolentry}{
        Jun 2018 -- May 2020
    }
        \textbf{Backend Engineer} | AvidXchange -- Charlotte, NC
    \end{twocolentry}

    \vspace{0.10cm}
    \begin{onecolentry}
        \begin{highlights}
            \item Developed Python APIs for transaction automation handling 200K+ daily operations, implementing observability with New Relic and Datadog to identify bottlenecks and support on-call production reliability.
            \item Led performance optimization initiative reducing month-end processing from 8 hours to 3 hours through comprehensive monitoring, profiling, and iterative improvements to caching and database queries.
            \item Optimized OCR processing backends with Redis caching and ML-based auto-coding, reducing invoice-to-payment delays by 25\% through systematic experimentation and measurement.
            \item Collaborated cross-functionally with operations teams, shadowing users to understand pain points and translating insights into technical solutions that improved vendor satisfaction scores by 35\%.
        \end{highlights}
    \end{onecolentry}

    \section{Education}

    \begin{twocolentry}{
        May 2018
    }
        \textbf{University of North Carolina at Charlotte}
    \end{twocolentry}

    \vspace{0.10cm}
    \begin{onecolentry}
        Master of Science, Computer Science (Focus: Machine Learning, Distributed Systems, NLP, CV)
    \end{onecolentry}

    \vspace{0.20cm}

    \begin{twocolentry}{
        May 2016
    }
        \textbf{University of North Carolina at Charlotte}
    \end{twocolentry}

    \vspace{0.10cm}
    \begin{onecolentry}
        Bachelor of Science, Computer Science (Major: Software Engineering)
    \end{onecolentry}

\end{document}
